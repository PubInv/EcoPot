 %% Copyright (C) 2021 by
%%   Robert L. Read <read.robert@gmail.com> and the Rice EcoPot team
%% Licensed under CC BY-ND-4.0:
%% https://creativecommons.org/licenses/by-nd/4.0/
%%
%% Attribution — You must give appropriate credit, provide a link tothe
%% license, and indicate if changes were made. You may do so in any
%% reasonable manner, but not in any way that suggests the licensor
%% endorses you or your use.

%% NoDerivatives — If you remix, transform, or build upon the material,
%% you may not distribute the modified material.

%% No additional restrictions — You may not apply legal terms or
%% technological measures that legally restrict others from doing
%% anything the license permits.


\documentclass{article}
\usepackage{hyperref}
\usepackage{amsmath}
\usepackage{amssymb}
\usepackage{mathtools}
\usepackage{draftwatermark}


\SetWatermarkText{DRAFT}
\SetWatermarkScale{6}
\SetWatermarkLightness{0.95}

\title{EcoPot: A More Efficient Pot for Wood Fires}

\author{
  TeamEcoPot of RiceUniversity \\
  Stephanie Ponce \\
  Christopher Fang \\
  San Robedee \\
  Chinwe \\
  Kaitlyn \\
  Sana \\
Robert L. Read
  \thanks{read.robert@gmail.com}
  email: \href{mailto:read.robert@gmail.com}{read.robert@gmail.com}
  }


\begin{document}

\maketitle
\begin{abstract}
\end{abstract}


\section{Introduction}

[Put a description of the globabl problem here; I suspect you
  have good information on that.]

This is a test citation to show what it looks like\cite{softrobotcalc};
this actual reference has no bearing on this work.

\section{Project History}

\section{Finned Pot History}

\section{User Research}

The team used contacts in Malawi to do direct, culture-specific user research,
even though many cultures cook on open fires with different techniques.

[Put survey and other information here!]

\section{Testing Strategy}

Our basic strategy was to design small, 100ml ``mini-pots'' with a CAD program to be
tested with ANSYS simulation software and to order small, affordable
3D printed
steel pots based on these designs.
The small tests could be tested relatively easily with simple test
apparatuses.


\section{Initial CAD Designs}

Initially the team developed 4 pot designs:
\begin{enumerate}
\item Control Pot
\item Full Radial Fins
\item Half-Radia Fins
\item Parallel Fins
\end{enumerate}

\section{Mini-pot Test Methodology}

Mini-pots were tested for boil time over the burner of a stove using
a standard clamp apparatus.

We also constructed a cardboard profile sheet and attempted to do
IR imaging on this sheet to verify our understanding of the fluid flow
and heat transfer [More needed here.]

\subsection{Mini-pot Test Data}

\section{Constructing large pots}

Based on the relative success of the min-pots, large, 12-cup
pots were constructed and tested over propate burners and
controlled wood fires.

\subsection{Rivets and Epoxy}

\subsection{JetBoil Fins}

\subsection{Large Pot Test Data}

\section{Rounded Bottom Designs}

Later the team designed a pot using the successful half-radial
approach that had a rounded (hemispherical) bottom.
The theory was that the Coanda effect would keep the hot gases
in contact with the pot. Both Ansys simulation and mini-pot flame
testing suggested this is actually true.
There is some reason to believe the rounded-bottom pots may be
a more efficient point in the design space than flat-bottomed pots,
although they may be more difficult to manufacture.

\section{Conclusions}

This work demonstrates that pots with heat-exchanging fins
can be more efficient than standard pots in some way.
If this translates to decreased fuel consumption without
inordinate additional expense or loss of durability,
these pots may ease fuel gathering burdens, cook faster,
and create less pollution than standard pots.

\bibliographystyle{acm}

\bibliography{softrobotmath}


\end{document}
